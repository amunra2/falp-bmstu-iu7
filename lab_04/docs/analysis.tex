\chapter{Практические задания}

\section{Задание 1}

\textbf{Условие:} чем принципиально отличаются функции \texttt{cons, list, append}?
Пусть \texttt{(setf lst1'( a b)); (setf lst2 '(c d))}. Каковы результаты вычисления следующих выражений?

\begin{lstlisting}
    (cons lst1 lst2)   ; ((A B) C D)
    (list lst1 lst2)   ; ((A B) (C D))
    (append lst1 lst2) ; (A B C D)
\end{lstlisting}


\section{Задание 2}

\textbf{Условие:} каковы результаты вычисления следующих выражений, и почему?

\begin{lstlisting}
    (reverse '()) ; Nil
    (last ()) ; Nil
    (reverse '(a)) ; (A)
    (last '(a)) ; (A)
    (reverse '((a b c))) ; ((ABC))
    (last '((a b c))) ; ((A B C))
\end{lstlisting}


\section{Задание 3}

\textbf{Условие:} написать, по крайней мере, два варианта функции, которая возвращает последний элемент своего списка-аргумента.


\begin{lstlisting}
    (defun get_last1 (lst) (first (reverse lst)))
    (defun get_last2 (lst) (if (eq (cdr lst) Nil) (car lst) (get_last2 (cdr lst))))
    (defun get_last3 (lst) (cond ((eq (cdr lst) Nil) (car lst)) (T (get_last3 (cdr lst)))))
\end{lstlisting}


\section{Задание 4}

\textbf{Условие:} написать, по крайней мере, два варианта функции, которая возвращает свой список-аргумент без последнего элемента.

\begin{lstlisting}
    (defun without_last1 (lst) (reverse (cdr (reverse lst))))
    (defun without_last2 (lst) (if (eq (cdr lst) Nil) () (cons (car lst) (without_last2 (cdr lst)))))
    (defun without_last3 (lst) (cond ((eq (cdr lst) Nil) ()) (T (cons (car lst) (without_last2 (cdr lst))))))
\end{lstlisting}


\section{Задание 5}

\textbf{Условие:} написать простой вариант игры в кости, в котором бросаются две правильные кости. Если сумма выпавших очков равна 7 или 11 -- выигрыш, если выпало (1,1) или (6,6) --- игрок имеет право снова бросить кости, во всех остальных случаях ход переходит ко второму игроку, но запоминается сумма выпавших очков. Если второй игрок не выигрывает абсолютно, то выигрывает тот игрок, у которого больше очков. Результат игры и значения выпавших костей выводить на экран с помощью функции print.


\begin{lstlisting}
    (defvar player1 "user")
    (defvar player2 "computer")
    
    (defvar dice1)
    (defvar dice2)
    (defvar tmp_dice)
    
    
    (defun roll_dice() 
        (+ (random 6) 1)
    )
    (defun roll_dices() 
        (list (roll_dice) (roll_dice))
    )
    
    (defun sum_points(dice) 
        (+ (car dice) (cadr dice))
    )
    
    (defun is_win(dice) 
        (cond ((= (sum_points dice) 7) T) ((= (sum_points dice) 11) T) )
    )
    
    (defun is_repeat(dice)
        (cond ((= (car dice) (cadr dice) 1) T) ((= (car dice) (cadr dice) 6) T) )
    )
    
    (defun is_player_won(result) 
        (= (cadr result) 1)
    )
    
    (defun print_res (player dice) 
        (format Nil "Win ~a, points = ~a, sum = ~a" player (car dice) (sum_points (car dice)))
    )
    
    
    (defun print_info (player dice) 
        (format T "Player: ~a, points = ~a, sum =  ~a" player dice (sum_points dice))
    )
    
    
    (defun player_move (player)
        (setf tmp_dice (roll_dices))
        (print_info player tmp_dice)
        (cond ((is_win tmp_dice) (list tmp_dice 1))
              ((is_repeat tmp_dice) (player_move player))
              (T (list tmp_dice 0)))
    )
    
    
    (defun start_game()
        (setf dice1 (player_move player1))
        (if (is_player_won dice1) (print_res player1 dice1)
            (and (setf dice2 (player_move player2))
            (cond ((is_player_won dice2) (print_res player2 dice2))
                  ((> (sum_points (car dice1)) (sum_points (car dice2))) (print_res player1 dice1))
                  ((< (sum_points (car dice1)) (sum_points (car dice2))) (print_res player2 dice2)) 
                  ((format Nil "Draw")) )))
    )
\end{lstlisting}





\chapter{Ответы на вопросы к лабораторной работе}

\section{Cинтаксическая форма и хранение программы в памяти}

В LISP формы представления программы и обрабатываемых ею данных одинаковы и представляются в виде S-выражений. Из-за этого программы могут обрабатывать и преобразовывать другие программы и даже самих себя. В процессе трансляции можно введенное и сформированное в результатевычислений выражение данных проинтерпретировать в качестве программы и непосредственно выполнить. 

Так как программа представляет собой S-выражение, в памяти она представлена либо как атом (5 указателей; форма представления атома в памяти (при этом память выделяется блоками)), либо списковой ячейкой (бинарныйузел; 2 указателя).


\section{Трактовка элементов списка}

Первый аргумент списка -- имя функции, а остальные -- аргументы этой функции (если при этом отсуствует блокировка вычислений).


\section{Порядок реализации программы}

Программа на языке Lisp -- S-выражение. Данное S-выражение передается функции \texttt{eval}, которая обрабатывает программу. Схема работы функции \texttt{eval} представлена ниже.

\imgHeight{70mm}{eval_work}{Работа функции eval}


\section{Способы определения функции}

Построить функцию можно с помощью Lambda-выражения (базисный способ).

Lambda-определение безымянной функции:

\begin{lstlisting}
    (lambda <Lambda-список> <форма>),
\end{lstlisting}

где Lambda-список -- список аргументов, а форма -- тело функции.

Lambda-вызов функции:

\begin{lstlisting}
    (<Lambda-выражение> <формальные параметры>)
\end{lstlisting}

Функции с именем. В таких функциях \texttt{defun} связывает символьный атом с Lambda-определением:

\begin{lstlisting}
    (defun f <Lambda-выражение>)
\end{lstlisting}

Упрощенное определение:

\begin{lstlisting}
    (defun f(x1, ..., xk) (<формы>))
\end{lstlisting}
