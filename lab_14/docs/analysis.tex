\chapter{Практические задания}

\textbf{Условие:} Создать базу знаний: <<ПРЕДКИ>>, позволяющую наиболее эффективным способом (за меньшее количество шагов, что обеспечивается меньшим количеством предложений БЗ -- правил), и используя  разные  варианты  (примеры) одного вопроса, определить(указать: какой вопрос для какого варианта):

\begin{enumerate}
	\item По имени субъекта определить всех его бабушек (предки 2-го колена).
	\item По имени субъекта определить всех его дедушек (предки 2-го колена).
	\item По  имени  субъекта  определить всех его  бабушек  и  дедушек  (предки  2-го колена).
	\item По имени субъекта определить его бабушку по материнской линии (предки 2-го колена).
	\item По имени субъекта определить его бабушку и дедушку по материнской линии (предки 2-го колена).
\end{enumerate}

Минимизировать количество правил и количество вариантов вопросов. Использовать конъюнктивные правила и простой вопрос. 

Для одного из вариантов вопроса и  конкретной бз составить  таблицу, отражающую конкретный порядок работы системы.

\subsubsection{Листинг программы}

\lstinputlisting{../src/lab_14.pro}


\subsubsection{Выполнение заданий}

Таблицы приложены в конце отчета
