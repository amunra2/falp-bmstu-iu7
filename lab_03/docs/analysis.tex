\chapter{Практические задания}

\section{Задание 1}

\textbf{Условие:} написать функцию, которая принимает целое число и возвращает первое четное число, не меньшее аргумента.

\begin{lstlisting}
	(defun find_even (num) (if (evenp num) num (+ num 1)))
	; (find_even 5) -> 6
	; (find_even 6) -> 6
\end{lstlisting}


\section{Задание 2}

\textbf{Условие:} написать функцию, которая принимает число и возвращает число того же знака, но с модулем на 1 больше модуля аргумента.

\begin{lstlisting}
	(defun inc_module (num) (if (< num 0) (- num 1) (+ num 1)))
	; (inc_module 5)  -> 6
	; (inc_module -5) -> -6
	; (inc_module 0)  -> 1
\end{lstlisting}


\section{Задание 3}

\textbf{Условие:} написать функцию, которая принимает два числа и возвращаетсписок из этих чисел, расположенный по возрастанию.


\begin{lstlisting}
	(defun mini_sort (a b) (if (> a b) (list b a) (list a b)))
	; (mini_sort 2 1) -> (2 1)
\end{lstlisting}


\section{Задание 4}

\textbf{Условие:} написать функцию, которая принимает три числа и возвращает Т только тогда, когда первое число расположено между вторым и третьим.

\begin{lstlisting}
	(defun is_middle (a b c) (cond ((and (> a b) (< a c)) T) ((and (> a c) (< a b)) T)))
\end{lstlisting}


\section{Задание 5}

\textbf{Условие:} каков результат вычисления следующих выражений?

\begin{lstlisting}
	(and 'fee 'fie 'foe) ; FOE
	(or Nil 'fie 'foe) ; FIE
	(and (equal 'abc 'abc) 'yes) ; YES
	(or 'fee 'fie 'foe) ; FEE
	(and nil 'fie 'foe) ; Nil
	(or (equal 'abc 'abc) 'yes) ; T
\end{lstlisting}


\section{Задание 6}

\textbf{Условие:} написать предикат, который принимает два числа-аргумента и возвращает Т, если первое число не меньше второго.

\begin{lstlisting}
	(defun predicate (a b) (>= a b))
	; (predicate 2 1) -> T
\end{lstlisting}


\section{Задание 7}

\textbf{Условие:} какой из следующих двух вариантов предиката ошибочен и почему?
\begin{lstlisting}
	(defun pred1 (x) (and (numberp x) (plusp x))) ; Ok
	(defun pred2 (x) (and (plusp x)(numberp x))) ; Если x не число, то функция plusp вернет ошибку <Not a number>
\end{lstlisting}


\section{Задание 8}

\textbf{Условие:} решить задачу 4, используя для ее решения конструкции \texttt{IF, COND, AND/OR}.

\begin{lstlisting}
	; if
	(defun between_if (a b c) (if (< b a c) (< b a c) (> b a c)))

	; cond 
	(defun between_cond (a b c) (cond ((< b a c) T) ((< c a b) T)))

	; and/or
	(defun between_andor (a b c) (or (< b a c) (< c a b))
\end{lstlisting}


\section{Задание 9}

\textbf{Условие:} Переписать функцию how-alike, приведенную в лекции и использующую \texttt{COND}, используятолько конструкции \texttt{IF, AND/OR}.

Исходная функция \texttt{how\_alike} представлена ниже.

\begin{lstlisting}
	(defun how_alike(x y)
	(cond ((or (= x y) (equal x y)) 'the_same)
		  ((and (oddp x) (oddp y)) 'both_odd)
		  ((and (evenp x) (evenp y)) 'both_even)
		  (t'difference)))
\end{lstlisting}

Функция \texttt{how\_alike} через \texttt{if} представлена ниже.

\begin{lstlisting}
	(defun how_alike(x y)
	(if (or (= x y) (equal x y)) 'the_same
	(if (and (oddp x) (oddp y)) 'both_odd
	(if (and (evenp x) (evenp y)) 'both_even 'difference))))
\end{lstlisting}

Функция \texttt{how\_alike} через \texttt{and/or} представлена ниже.

\begin{lstlisting}
	(defun how_alike(x y)
	(or (and (or (= x y) (equal x y)) 'the_same)
		(and (and (oddp x) (oddp y)) 'both_odd)
		(and (and (evenp x) (evenp y)) 'both_even)
		'difference))
\end{lstlisting}


\chapter{Ответы на вопросы к лабораторной работе}

\section{Базис Lisp.}

\textbf{Базис языка} -- минимальный набор конструкций языка и структур данных, с помощью которых можно решить любую задачу.

Базис состоит из:
\begin{enumerate}
    \item атомы и структуры (представляющиеся бинарными узлами);
    \item базовые (несколько) функций и функционалов: встроенные примитивные функции \texttt{(atom, eq, cons, car, cdr)}; специальные функции и функционалы \texttt{(quote, cond, lambda, eval, apply, funcall)}.
\end{enumerate}


\section{Классификация функций.}

Функции в Lisp классифицируют следующим образом:

\begin{itemize}
    \item <<чистые>> математические функции;
    \item рекурсивные функции;
    \item специальные функции --- формы (от вызова к вызову может меняться количество аргументов, или обрабатываться по-разному);
    \item псевдофункции (создают эффект на внешнем устройстве);
    \item функции с вариативными значениями, из которых выбирается 1;
    \item функционалы (в качестве аргумента -- функцмя, или возаращает в качестве результата функцию).
\end{itemize}

По назначению функции разделяются следующим образом:

\begin{enumerate}
    \item конструкторы --- создают значение (\texttt{cons}, например);
    \item селекторы --- получают доступ по адресу (\texttt{car}, \texttt{cdr});
    \item предикаты --- возвращают \texttt{Nil}, \texttt{T}.
\end{enumerate}

\section{Способы создания функций.}

Построить функцию можно с помощью Lambda-выражения (базисный способ)

Lambda-определение безымянной функции:

\begin{lstlisting}
	(lambda <Lambda-список> <форма>),
\end{lstlisting}

где Lambda-список -- список аргументов, а форма -- тело функции.

Lambda-вызов функции:

\begin{lstlisting}
	(<Lambda-выражение> <формальные параметры>)
\end{lstlisting}

Функции с именем. В таких функциях \texttt{defun} связывает символьный атом с Lambda-определением:

\begin{lstlisting}
	(defun f <Lambda-выражение>)
\end{lstlisting}

Упрощенное определение:

\begin{lstlisting}
	(defun f(x1, ..., xk) (<формы>))
\end{lstlisting}


\section{Работа функций Cond, if, and/or.}

\texttt{cond} -- средство разветвления вычислений. Условное выражение вычисляется по следующим правилам:

\begin{enumerate}
	\item Последовательно вычсляются условия ветвей до тех пор, пока не встретится выражение-форма, значение которой отлично от Nil.
	\item Затем вычисляется выражение соответсвующей ветви и его значение возвращается в качестве значения функции.
	\item Если все условия имеют значение Nil, то значением условного выражения становится Nil.
\end{enumerate}

\begin{lstlisting}
	(cond (p1 e1) (p2 e2) ... (pn en))
\end{lstlisting}


\texttt{if} -- макрофункция. Если значение выражения \texttt{C} отлично от \texttt{Nil}, то вычисляется выражение \texttt{E1}, иначе значение \texttt{E2}.

\begin{lstlisting}
	(if C E1 E2)
\end{lstlisting}

\texttt{or} -- функция-дизъюнкции.

\begin{lstlisting}
	(or e1 e2 ... en)
\end{lstlisting}

При  выполнении вызова последовательно  вычисляются  аргументы ei (слева направо) -- до тех пор, пока не встретится значение ei, отличное от Nil.  В этом случае  вычисление  прерывается  изначение  функции  равно значению этого ei. Если  же  вычислены значения всех  аргументов ei, и оказалось,  что они    равны Nil,  то  результирующее  значение  функции равно Nil.


\texttt{and} -- функция-конъюнкции.

\begin{lstlisting}
	(and e1 e2 ... en)
\end{lstlisting}

При  вычислении этого  функционального обращения последовательно слева направо вычисляются аргументы функции ei -- до тех пор, пока не встретится значение, равное Nil. В этом случае вычисление прерывается и значение функции равно Nil. Если же были вычислены все значения ei и оказалось,  что все  они отличны  от Nil,  то  результирующим значением функции and будет значение последнего выражения en. 
