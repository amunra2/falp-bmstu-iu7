\chapter{Практические задания}

\section{Лабораторная работа 12 (часть 1)}

\textbf{Условие:} Составить программу, т.е. модель предметной области -- базу знаний, объединив в ней информацию-знания:

\begin{itemize}
	\item \textbf{«Телефонный  справочник»}: Фамилия,  №тел, Адрес -- структура(Город, Улица, №дома, №квартиры);
	\item \textbf{«Автомобили»}: Фамилия\_владельца, Марка, Цвет, Стоимость, и др.;
	\item \textbf{«Вкладчики банков»}: Фамилия, Банк, счет, сумма, др.
\end{itemize} 

Владелец может иметь несколько телефонов, автомобилей, вкладов (Факты). Используя правила, обеспечить возможность поиска:

\begin{enumerate}
	\item \begin{enumerate}
		\item По № телефона найти: Фамилию, Марку автомобиля, Стоимость автомобиля (может быть несколько);
		\item Используя сформированное в предыдущем пункте правило, по № телефона найти: только Марку автомобиля (автомобилей может быть несколько).
	\end{enumerate}

	\item Используя простой, не составной вопрос: по Фамилии(уникальна в городе, но в разных  городах  есть  однофамильцы) и  Городу  проживания  найти: Улицу проживания, Банки, в которых есть вклады и № телефона.
\end{enumerate}

\subsubsection{Листинг программы}

\lstinputlisting{../src/lab12_01.pro}


\subsubsection{Выполнение заданий}

Таблицы приложены в конце отчета

\section{Лабораторная работа 12 (часть 2)}

\textbf{Условие:} Для базы знаний из Части 1 лабораторной работы 12, используя конъюнктивное правило и простой вопрос, обеспечить возможность поиска: 

По Марке и Цвету автомобиля найти Фамилию, Город, Телефон и Банки, в которых владелец автомобиля имеет вклады. Лишней информации не находить и не передавать!!! Владельцев может быть несколько (не более 3-х), один и ни одного. 


\begin{enumerate}
	\item Для каждого из трех вариантов словесно подробно описать порядок формирования ответа (в виде таблицы). При этом, указать – отметить моменты очередного запускаалгоритма унификации и полный результат его работы. Обосновать следующий шаг работы системы. Выписать унификаторы – подстановки. Указать моменты, причины и результат отката, если он есть.
	\item Для случая нескольких владельцев (2-х): приведите примеры (таблицы) работы системы при разных порядках следования в БЗ процедур, и знаний в них: («Телефонный справочник», «Автомобили», «Вкладчики банков», или: «Автомобили», «Вкладчики банков», «Телефонный справочник»). Сделайте вывод:Одинаковы ли: множество работ и объем работ в разных случаях? 
	\item Оформите 2 таблицы, демонстрирующие порядок работы алгоритма унификации вопроса и подходящего заголовка правила (для двух случаев из пункта 2) и укажите результаты его работы: ответ и побочный эффект.
\end{enumerate}


\subsubsection{Листинг программы}

\lstinputlisting{../src/lab12_02.pro}


\subsubsection{Выполнение заданий}

Таблицы приложены в конце отчета
