\chapter{Практические задания}

Практические задания приложены к отчету.

\chapter{Ответы на вопросы к лабораторной работе}

\section{Элементы языка: определение, синтаксис, представление в памяти.}

\subsection{Определение}

Вся информация (данные и программы) в Lisp представляются в виде символьных выражений -- S-выражений. По определению:

\begin{lstlisting}
	S-выражение ::= <атом> | <точечная пара> 
\end{lstlisting}

\textbf{Атомами} могут являться:
\begin{enumerate}
	\item Символы -- синтаксически представляется как набор букв и цифр, начинающийся с буквы.
	\item Специальные символы -- \{T, Nil\}:
	\begin{itemize}
		\item T -- обозначает логическое значение <<истина>>, истинным значением является все, отличное от Nil;
		\item Nil -- обозначает логическое значение <<ложь>>, также обозначает пустой список (записи Nil и () эквивалентны);
	\end{itemize}
	\item Самоопределимые атомы -- натуральные, дробные и вещественные числа, а также строки, заключенные в двойные апострофы.
\end{enumerate}


\textbf{Точечная пара} - (атом1 . атом2). Строится с помощью бинарного узла, в котором левая и правая части равноправны и хранят указатели на атомы (вместо атомов могут быть более сложные структуры).   

\begin{lstlisting}
	Точечная пара ::= (<атом>.<атом>) |
					  (<атом>.<точечная пара>) |
					  (<точечная пара>.<атом>) |
					  (<точечная пара>.<точечная пара>)	
\end{lstlisting}


\textbf{Список} - динамическая структура данных, которая может быть
пустой или непустой. Если она не пустая, то состоит из двух элементов:

\begin{enumerate}
	\item Голова -- любая структура.
	\item Хвост -- список.
\end{enumerate}


\subsection{Синтаксис}

Любая структура заключается в круглые скобки (A.B) -- точечная пара, (A) -- список из одного элемента, при этом пустой список: Nil или ().

Записать список можно следующим образом: (A. (B. (C()))) или (A B C). 

Элементы списка, в свою очередь, могут быть списками, например -- (A (B C) (D (E))). То есть, наличие скобок является признаком структуры -- списка или точечной пары.

\subsection{Представление в памяти}

\begin{enumerate}
	\item (A . B) -- точечная пара.

	\imgHeight{40mm}{memory_para}{Представление в памяти (A . B)}

	\item (A B) -- список из двух элементов.

	\imgHeight{40mm}{memory_list}{Представление в памяти (A B)}

\end{enumerate}


\section{Особенности языка Lisp. Структура программы. Символ апостроф.}

Особенности языка Lisp.

\begin{enumerate}
	\item В LISP используется символьная обработка.
	\item Программа может быть представлена в виде данных, поэтому программа может изменять сама себя.
	\item Нет типизации (бестиповый язык).
	\item Память выделяется блоками. LISP сам распределяет память.
	\item Программы, написанные на Лисп, представляются в виде списков.
\end{enumerate}


Символ апостроф (quote, <<'>>) -- блокирует вычисление своего аргумента. В качестве своего значения выдаёт сам аргумент, не вычисляя его. Перед константами – числами и  атомами T, Nil можно не ставить апостроф. Пример:

\begin{lstlisting}
	'(CAR(A B C D)) => (CAR(A B C D))
\end{lstlisting}

\section{Базис языка Lisp. Ядро языка.}

\textbf{Базис языка} -- минимальный набор конструкций языка и структур данных, с помощью которых можно решить любую задачу.

Базис состоит из:
\begin{enumerate}
    \item структур, атомов;
    \item встроенных (примитивных) функций (\texttt{car}, \texttt{cdr});
    \item специальных функций, управляющих обработкой структур, представляющих вычислимые выражения (\texttt{quote}).
\end{enumerate}

\textbf{Ядро} -- основные действия, которые наиболее часто используются. Ядро шире, чем базис.
