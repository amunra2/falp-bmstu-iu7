\chapter{Практические задания}

\section{Задание 1}

\textbf{Условие:} напишите функцию, которая уменьшает на 10 все числа из списка-аргумента этой функции.

\begin{lstlisting}
(defun minus10 (lst)
    (mapcar #'(lambda (elem) (cond ((numberp elem) (- elem 10))
                                   ((listp elem) (minus10 elem)) 
                                   (T elem))
                             ) lst)
)
\end{lstlisting}


\section{Задание 2}

\textbf{Условие:} напишите функцию, которая умножает на заданное число-аргумент все числа из заданного списка-аргумента, когда
\begin{itemize}
    \item все элементы списка --- числа;
    \item элементы списка -- любые объекты.
\end{itemize}


\begin{lstlisting}
; а)
(defun mul-num (lst num)
    (mapcar #'(lambda (elem) (* elem num)) lst)
)

; б)
(defun mul-num (lst num)
    (mapcar #'(lambda (elem) (cond ((numberp elem) (* elem num))
                                   ((listp elem) (mul-num elem num))
                                   (T elem)
                             )
              ) lst
    )
)
\end{lstlisting}


\section{Задание 3}

\textbf{Условие:} написать функцию, которая по своему списку-аргументу \texttt{lst} определяет, является ли он палиндромом (то есть равны ли \texttt{lst} и \texttt{(reverse lst)}).



\begin{lstlisting}
(defun palindrom-func (lst rev_lst)
    (mapcar #'(lambda (elem1 elem2) (equal elem1 elem2)) lst rev_lst)
)


(defun palindrom (lst)
    (reduce #'(lambda (elem1 elem2) (and elem1 elem2)) 
                            (palindrom-func lst (reverse lst)))
)
\end{lstlisting}


\section{Задание 4}

\textbf{Условие:} написать предикат \texttt{set-equal}, который возвращает \texttt{t}, если два его множества-аргумента содержат одни и те же элементы, порядок которых не имеет значения.

\begin{lstlisting}
(defun my-subsetp (set1 set2)
    (reduce #'(lambda (prev_res1 elem1)
                (and prev_res1 (reduce #'(lambda (prev_res2 elem2)
                    (or prev_res2 (equal elem1 elem2))
                ) set2 :initial-value Nil))
            ) set1 :initial-value T)
)

(defun set-equal (lst1 lst2)
    (and (my-subsetp lst1 lst2) (my-subsetp lst2 lst1))
)
\end{lstlisting}


\section{Задание 5}

\textbf{Условие:} написать функцию, которая получает как аргумент список чисел, а возвращает список квадратов этих чисел в том же порядке.

\begin{lstlisting}
(defun quad (lst)
    (mapcar #'(lambda (elem) (* elem elem)) lst)
)
\end{lstlisting}


\section{Задание 6}

\textbf{Условие:} напишите функцию, \texttt{select-between}, которая из списка-аргумента из 5 чисел выбирает только те, которые расположены между двумя указанными границами-аргументами и возвращает их в виде списка (упорядоченного по возрастанию списка чисел (+ 2 балла)).

\begin{lstlisting}
(defun select-between (lst a b)
    (remove-if #'(lambda (elem) (not (or (> a elem b) (< a elem b)))) lst)
)
\end{lstlisting}


\section{Задание 7}

\textbf{Условие:} написать функцию, вычисляющую декартово произведение двух своих списков-аргументов.

\begin{lstlisting}
(defun decart (lst1 lst2)
    (mapcan #'(lambda (elem1) (
        mapcar #'(lambda (elem2) (list elem1 elem2)) lst2
    )) lst1)
)
\end{lstlisting}


\section{Задание 8}

\textbf{Условие:} почему так реализовано \texttt{reduce}, в чем причина?

Все дело в параметре \texttt{initial-value}, который указывается, а затем применяется функция. По умолчанию, для операции сложения этот параметр равен 0, а для опрации умножения -- 1. 
Поэтому:
\begin{verbatim}
    (reduce #'+ ()) ; -> 0
\end{verbatim}

Случай ниже выдаст ошибку:
\begin{verbatim}
    (reduce #'+ 0)   ; -> Error
\end{verbatim}

Ошибка:
\texttt{The value 0 is not of type SEQUENCE}


\section{Задание 9}

\textbf{Условие:} пусть \texttt{list-of-list} список, состоящий из списков. Написать функцию, которая вычисляет сумму длин всех элементов \texttt{list-of-list}, т.е. например для аргумента \texttt{((1 2) (3 4)) -> 4}.


\begin{lstlisting}
(defun deep-length (lst)
    (reduce '+ (mapcar #'(lambda (elem) 
                            (cond ((listp elem) (deep-length elem))
                                  (T 1)
    )) lst))
)
\end{lstlisting}
