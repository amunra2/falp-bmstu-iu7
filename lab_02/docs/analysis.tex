\chapter{Практические задания}

\section{Задание 1}

\textbf{Условие:} составить диаграмму вычисления выражений.

Решение приложено к отчету.


\section{Задание 2}

\textbf{Условие:} написать функцию, вычисляющую гипотенузу прямоугольного треугольника по заданным катетам и составить диаграмму её вычисления.

\begin{lstlisting}
	решение
\end{lstlisting}

Диаграмма приложена к отчету.


\section{Задание 3}

\textbf{Условие:} написать функцию, вычисляющую объем параллелепипеда по 3-м его сторонам, и составить диаграмму ее вычисления.

\begin{lstlisting}
	решение
\end{lstlisting}

Диаграмма приложена к отчету.


\section{Задание 4}

\textbf{Условие:} каковы результаты вычисления следующих выражений? (объяснить возможную ошибку и варианты ее устранения)

\begin{lstlisting}
	решение
\end{lstlisting}


\section{Задание 5}

\textbf{Условие:} написать функцию \texttt{longer\_then} от двух списков-аргументов, которая возвращает Т, если первый аргумент имеет большую длину.

\begin{lstlisting}
	решение
\end{lstlisting}


\section{Задание 6}

\textbf{Условие:} каковы результаты вычисления следующих выражений?

\begin{lstlisting}
	решение
\end{lstlisting}


\section{Задание 7}

\textbf{Условие:} дана функция \texttt{(defun mystery (x) (list (second x) (first x)))}.Какие результаты вычисления следующих выражений? 

\begin{lstlisting}
	решение
\end{lstlisting}


\section{Задание 8}

\textbf{Условие:} написать функцию, которая переводит температуру в системе Фаренгейтатемпературу по Цельсию \texttt{(defun f-to-c (temp)...)}. 

Формулы: $c = \frac{5}{9} \cdot (f -- 320)$; $f= \frac{5}{9} \cdot c + 32.0$. Как бы назывался роман Р.Брэдбери "+451 по Фаренгейту" в системе по Цельсию?

\begin{lstlisting}
	решение
\end{lstlisting}


\section{Задание 9}

\textbf{Условие:} Что получится при вычисления каждого из выражений?

\begin{lstlisting}
	решение
\end{lstlisting}


\chapter{Ответы на вопросы к лабораторной работе}

\section{Базис Lisp.}

\textbf{Базис языка} -- минимальный набор конструкций языка и структур данных, с помощью которых можно решить любую задачу.

Базис состоит из:
\begin{enumerate}
    \item структур, атомов;
    \item встроенных (примитивных) функций (\texttt{car}, \texttt{cdr});
    \item специальных функций, управляющих обработкой структур, представляющих вычислимые выражения (\texttt{quote}).
\end{enumerate}

\textbf{Ядро} -- основные действия, которые наиболее часто используются. Ядро шире, чем базис.

% \begin{lstlisting}
% \end{lstlisting}

\section{Классификация функций.}

TODO

\section{Способы создания функций.}

TODO

\section{Функции CAR, CDR.}

TODO

\section{Назначение и отличие в работе Cons и List.}

TODO
