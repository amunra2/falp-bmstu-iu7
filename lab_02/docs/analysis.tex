\chapter{Практические задания}

\section{Задание 1}

\textbf{Условие:} составить диаграмму вычисления выражений.

Решение приложено к отчету.


\section{Задание 2}

\textbf{Условие:} написать функцию, вычисляющую гипотенузу прямоугольного треугольника по заданным катетам и составить диаграмму её высчиления.

\begin{lstlisting}
	(defun hypotenuse (a b) (sqrt (+ (* a a) (* b b))))
	; (HYPOTENUSE 3 4) -> 5.0
\end{lstlisting}

Диаграмма приложена к отчету.


\section{Задание 3}

\textbf{Условие:} написать функцию, вычисляющую объем параллелепипеда по 3-м его сторонам, и составить диаграмму ее вычисления.

\begin{lstlisting}
	(defun volume (a b c) (* a (* b c)))
	; (volume 4 4 4) -> 64
\end{lstlisting}

Диаграмма приложена к отчету.


\section{Задание 4}

\textbf{Условие:} каковы результаты вычисления следующих выражений? (объяснить возможную ошибку и варианты ее устранения)

\begin{lstlisting}
	(list 'a c) ; The variable C is unbound.
	(cons 'a (b c)) ; The variable C is unbound.
	(cons 'a '(b c)) ; (A B C)
	(caddy (1 2 3 4 5)) ; Undefined function: CADDY
	(cons 'a 'b 'c) ; invalid number of arguments: 3
	(list 'a (b c)) ; The variable C is unbound.
	(list a '(b c)) ; The variable A is unbound.
	(list (+ 1 '(length '(1 2 3)))) ; The value (LENGTH '(1 2 3)) is not of type NUMBER
\end{lstlisting}

Исправления ошибок представлены ниже.

\begin{lstlisting}
	(list 'a 'c) ; (A C)
	(cons 'a '(b c)) ; (A B C)
	(cons 'a '(b c)) ; (A B C)
	(caddr '(1 2 3 4 5)) ; 3
	(cons 'a 'b) ; (A B)
	(list 'a '(b c)) ; (A (B C))
	(list 'a '(b c)) ; (A (B C))
	(list (+ 1 (length '(1 2 3)))) ; (4)
\end{lstlisting}


\section{Задание 5}

\textbf{Условие:} написать функцию \texttt{longer\_then} от двух списков-аргументов, которая возвращает Т, если первый аргумент имеет большую длину.

\begin{lstlisting}
	(defun longer (a b) (cond ((> (length a) (length b)) T) (Nil)))
	; (longer '(a b) '(a)) -> T
	; (longer '(a) '(a)) -> Nil
	; (longer '(a) '(a b)) -> Nil
\end{lstlisting}


\section{Задание 6}

\textbf{Условие:} каковы результаты вычисления следующих выражений?

\begin{lstlisting}
	(cons 3 (list 5 6)) ; (3 5 6)

	(list 3 'from 9 'lives (-9 3)) ; illegal function call
	; Fix: (list 3 'from 9 'lives '(-9 3)) -> (3 FROM 9 LIVES (-9 3))

	(+ (length for 2 too))(car '(21 22 23))) ; The variable FOR is unbound.
	; Fix: (+ (length '(for 2 too))(car '(21 22 23))) ; 24

	(cdr '(cons is short for ans)) ; (IS SHORT FOR ANS)

	(car (list one two)) ; The variable ONE is unbound.
	; Fix: (car (list 'one 'two)) -> ONE

	(cons 3 '(list 5 6)) ; (3 LIST 5 6)

	(car (list 'one 'two)) ; ONE
\end{lstlisting}


\section{Задание 7}

\textbf{Условие:} дана функция \texttt{(defun mystery (x) (list (second x) (first x)))}.Какие результаты вычисления следующих выражений? 

\begin{lstlisting}
	(mystery (one two)) ; The variable TWO is unbound.
	; Fix: (mystery '(one two)) -> (TWO ONE)
	(mystery (one 'two)) ; The function COMMON-LISP-USER::ONE is undefined.
	(mystery (last one two)) ; The variable ONE is unbound.
	(mystery free) ; The variable FREE is unbound.
\end{lstlisting}


\section{Задание 8}

\textbf{Условие:} написать функцию, которая переводит температуру в системе Фаренгейта температуру по Цельсию \texttt{(defun f-to-c (temp)...)}. 

Формулы: $c = \frac{5}{9} \cdot (f -- 32.0)$; $f= \frac{9}{5} \cdot c + 32.0$. 

Как бы назывался роман Р.Брэдбери <<+451 по Фаренгейту>> в системе по Цельсию?

\begin{lstlisting}
	(defun f-to-c (temp) (* 5/9 (- temp 32.0)))
	; (f-to-c 451) -> 232.77779
\end{lstlisting}

Роман бы назывался <<+232 по Цельсию>>.


\section{Задание 9}

\textbf{Условие:} Что получится при вычисления каждого из выражений?

\begin{lstlisting}
	(list 'cons t nil) ; (CONS T NIL)
	(eval (list 'cons t NIL)) ; (T)
	(eval (eval (list 'cons t NIL))) ; The function COMMON-LISP:T is undefined.
	(apply #cons ''(t NIL)) ; illegal complex number format: #CONS
	; Fix: (apply #'cons '(t NIL)) -> (T)
	(eval Nil) ; Nil
	(list 'eval Nil) ; (EVAL NIL)
	(eval (list 'eval Nil)) ; NIL
\end{lstlisting}


\section{Дополнительное задание 1}

\textbf{Условие:} Написать функцию, вычисляющую катет по заданной гипотенузе и другому катету прямоугольного треугольника, и составить диаграмму ее вычисления.

\begin{lstlisting}
	(defun catet (b c) (sqrt (- (* c c) (* b b))))
	; (catet 4 5) -> 3.0
\end{lstlisting}

Диаграмма приложена к отчету.


\section{Дополнительное задание 2}

\textbf{Условие:} Написать функцию, вычисляющую площадь трапеции по ее основаниям и высоте, и составить диаграмму ее вычисления.

\begin{lstlisting}
	(defun catet (b c) (sqrt (- (* c c) (* b b))))
	; (catet 4 5) -> 3.0
\end{lstlisting}

Диаграмма приложена к отчету.




\chapter{Ответы на вопросы к лабораторной работе}

\section{Базис Lisp.}

\textbf{Базис языка} -- минимальный набор конструкций языка и структур данных, с помощью которых можно решить любую задачу.

Базис состоит из:
\begin{enumerate}
    \item атомы и структуры (представляющиеся бинарными узлами);
    \item базовые (несколько) функций и функционалов: встроенные примитивные функции \texttt{(atom, eq, cons, car, cdr)}; специальные функции и функционалы \texttt{(quote, cond, lambda, eval, apply, funcall)}.
\end{enumerate}


\section{Классификация функций.}

Функции в Lisp классифицируют следующим образом:

\begin{itemize}
    \item <<чистые>> математические функции;
    \item рекурсивные функции;
    \item специальные функции --- формы (от вызова к вызову может меняться количество аргументов, или обрабатываться по-разному);
    \item псевдофункции (создают эффект на внешнем устройстве);
    \item функции с вариативными значениями, из которых выбирается 1;
    \item функционалы (в качестве аргумента -- функцмя, или возаращает в качестве результата функцию).
\end{itemize}

По назначению функции разделяются следующим образом:

\begin{enumerate}
    \item конструкторы --- создают значение (\texttt{cons}, например);
    \item селекторы --- получают доступ по адресу (\texttt{car}, \texttt{cdr});
    \item предикаты --- возвращают \texttt{Nil}, \texttt{T}.
\end{enumerate}

\section{Способы создания функций.}

Построить функцию можно с помощью Lambda-выражения (базисный способ)

Lambda-определение безымянной функции:

\begin{lstlisting}
	(lambda <Lambda-список> <форма>),
\end{lstlisting}

где Lambda-список -- список аргументов, а форма -- тело функции.

Lambda-вызов функции:

\begin{lstlisting}
	(<Lambda-выражение> <формальные параметры>)
\end{lstlisting}

Функции с именем. В таких функциях \texttt{defun} связывает символьный атом с Lambda-определением:

\begin{lstlisting}
	(defun f <Lambda-выражение>)
\end{lstlisting}

Упрощенное определение:

\begin{lstlisting}
	(defun f(x1, ..., xk) (<формы>))
\end{lstlisting}


\section{Функции CAR, CDR.}

\texttt{car} -- функция получения первого элемента точечной пары, \texttt{cdr} -- функция получения второго элемента точечной пары.

Примеры:

\begin{center}
    \begin{tabular}{ |c|c|c| } 
        \hline
            S-выражение & Результат выполнения \texttt{car} & Результат выполнения \texttt{cdr} \\ 
        \hline
        \hline
            (A . B) & A & B \\ 
        \hline
            ((A . B) . C) & (A . B) & C \\ 
		\hline
			(A . (B . C)) & A & (B . C) \\ 
        \hline
    \end{tabular}
\end{center}


\section{Назначение и отличие в работе Cons и List.}

\texttt{cons} -- имеет фиксированное количество аргументов (два). 
В случае, когда аргументами являются атомы создает точечную пару. В случает, когда первый аргумент атом а второй список, атом становится головой, а второй аргумент (список) становится хвостом. 

\begin{lstlisting}[language=Lisp]
(cons 'a 'b) 	   ; (A . B)
(cons 'a '(b c d)) ; (A B C D)
(cons 'a 'b 'c)    ; Error (invalid number of arguments: 3)
\end{lstlisting}

\imgHeight{40mm}{memory_para}{Результат работы (cons(A B))}


\texttt{list} -- не имеет фиксированное количество аргументов. 
Создает список, у которого голова - это первый аргумент, хвост - все остальные аргументы.

\begin{lstlisting}[language=Lisp]
(list 'a 'b)    ; (A B)
(cons 'a 'b 'c) ; (A B C)
\end{lstlisting}

\imgHeight{40mm}{memory_list}{Результат работы (list(A B))}
