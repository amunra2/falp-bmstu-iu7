\chapter{Практические задания}

\section{Задание 1}

\textbf{Условие:} написать хвостовую рекурсивную функцию \texttt{my-reverse}, которая развернет верхний уровень своего списка-аргумента \texttt{lst}.

\begin{lstlisting}    
(defun move-to (lst result)
    (cond ((null lst) result)
            (T (move-to (cdr lst) (cons (car lst) result)))
    )
)

(defun my-reverse (lst)
    (move-to lst ())
)
\end{lstlisting}


\section{Задание 2}

\textbf{Условие:} написать функцию, которая возвращает первый элемент списка-аргумента, который сам является непустым списком.


\begin{lstlisting}
(defun first-lst (lst)
    (cond ((null lst) Nil)
          ((and (not (null (car lst))
                (listp (car lst)))) (caar lst))
          (T (first-lst (cdr lst)))
    )
)
\end{lstlisting}


\section{Задание 3}

\textbf{Условие:} написать функцию, которая выбирает из заданного списка только те числа, которые больше 1 и меньше 10.


\begin{lstlisting}
(defun check-border (num a b)
    (or (> a num b) (< a num b))
)

(defun select-between-rec (lst a b result)
    (cond ((null lst) result)
          ((and (numberp (car lst)) 
                  (check-border (car lst) a b)) 
                      (select-between-rec (cdr lst) a b (cons (car lst) result)))
          (T (select-between-rec (cdr lst) a b result))
    )
)

(defun select-between (lst)
    (select-between-rec lst 1 10 ())
)
\end{lstlisting}


\section{Задание 4}

\textbf{Условие:} напишите рекурсивную функцию, которая умножает на заданное число-аргумент все числаиз заданного списка-аргумента, когда
\begin{itemize}
    \item все элементы списка --- числа;
    \item элементы списка -- любые объекты.
\end{itemize}


\begin{lstlisting}
(defun move-to (lst result)
    (cond ((null lst) result)
          (T (move-to (cdr lst) (cons (car lst) result)))
    )
)

(defun my-reverse (lst)
    (move-to lst ())
)

; а)
(defun mul-num-rec (lst num result)
    (cond ((null lst) result)
          (T (mul-num-rec (cdr lst) num (cons (* (car lst) num) result)))
    )
)

(defun mul-num (lst num)
    (my-reverse (mul-num-rec lst num ()))
)

; б)
(defun mul-num-rec (lst num result)
    (cond ((null lst) result)
          ((symbolp (car lst)) 
                (mul-num-rec (cdr lst) num (cons (car lst) result)))
          ((numberp (car lst)) 
                (mul-num-rec (cdr lst) num (cons (* (car lst) num) result)))
          (T (mul-num-rec (cdr lst) num 
                (cons (my-reverse (mul-num-rec (car lst) num ())) result)))
    )
)

(defun mul-num (lst num)
    (my-reverse (mul-num-rec lst num ()))
)
\end{lstlisting}


\section{Задание 5}

\textbf{Условие:} напишите функцию, \texttt{select-between}, которая из списка-аргумента, содержащего только числа, выбирает только те, которые расположены между двумя указанными границами-аргументами и возвращает их в виде списка (упорядоченного по возрастанию списка чисел (+ 2 балла)).

\begin{lstlisting}
(defun check-border (num a b)
    (or (> a num b) (< a num b))
)

(defun select-between-rec (lst a b result)
    (cond ((null lst) result)
          ((and (numberp (car lst)) 
                (check-border (car lst) a b)) 
                    (select-between-rec (cdr lst) a b (cons (car lst) result)))
          (T (select-between-rec (cdr lst) a b result))
    )
)

(defun select-between (lst a b)
    (select-between-rec lst a b ())
)
\end{lstlisting}


\section{Задание 6}

\textbf{Условие:} написать рекурсивную версию (с именем rec-add) вычисления суммы чисел заданного списка:
\begin{itemize}
    \item одноуровнего смешанного;
    \item структурированного.
\end{itemize}

\begin{lstlisting}
; a)
(defun rec-add-rec (lst result)
    (cond ((null lst) result)
          ((numberp (car lst)) (rec-add-rec (cdr lst) (+ (car lst) result)))
          (T (rec-add-rec (cdr lst) result))
    )
)

(defun rec-add (lst)
    (rec-add-rec lst 0)
)

; б)
(defun rec-add-rec (lst result)
    (cond ((null lst) result)
          ((symbolp (car lst)) 
                (rec-add-rec (cdr lst) result))
          ((numberp (car lst)) 
                (rec-add-rec (cdr lst) (+ (car lst) result)))
          (T (rec-add-rec (cdr lst) (rec-add-rec (car lst) result)))
    )
)

(defun rec-add (lst)
    (rec-add-rec lst 0)
)    
\end{lstlisting}


\section{Задание 7}

\textbf{Условие:} написать рекурсивную версию с именем \texttt{recnth} функции \texttt{nth}.

\begin{lstlisting}
(defun recnth (lst n)
    (cond ((null lst) Nil)
          ((= n 0) (car lst))
          (T (recnth (cdr lst) (- n 1)))
    )
)
\end{lstlisting}


\section{Задание 8}

\textbf{Условие:} написать рекурсивную функцию \texttt{allodd}, которая возвращает \texttt{t}, когда все элементы списка нечетные.

\begin{lstlisting}
(defun allodd (lst)
    (cond ((null lst) T)
          ((oddp (car lst)) (allodd (cdr lst)))
          (T Nil)
    )
)   
\end{lstlisting}


\section{Задание 9}

\textbf{Условие:} написать рекурсивную функцию, которая возвращает первое нечетное число из списка (структурированного), возможно создавая некоторые вспомогательные функции.


\begin{lstlisting}
(defun first-odd (lst)
    (cond ((null lst) Nil)
          ((and (numberp (car lst)) 
                (oddp (car lst))) (car lst))
          ((listp (car lst)) 
                (or (first-odd (car lst)) (first-odd (cdr lst))))
          (T (first-odd (cdr lst)))
    )
)
\end{lstlisting}



\section{Задание 10}

\textbf{Условие:} используя cons-дополняемую рекурсию с одним тестом завершения, написать функцию которая получает как аргумент список чисел, а возвращает список квадратов этих чисел в том же порядке.


\begin{lstlisting}
(defun quad (lst)
    (cond ((null lst) Nil)
          (T (cons (* (car lst) (car lst)) (quad (cdr lst)) ))
    )
)
\end{lstlisting}
