\chapter{Практические задания}

\textbf{Условие:} В одной программе написать правила, позволяющие найти 

\begin{enumerate}
	\item Максимум из двух чисел
	\begin{enumerate}
		\item Без использования отсечения.
		\item С использованием отсечения.
	\end{enumerate}

	\item Максимум из трех чисел
	\begin{enumerate}
		\item Без использования отсечения.
		\item С использованием отсечения.
	\end{enumerate}
\end{enumerate}

Убедиться в правильности результатов. Для каждого случая пункта 2 обосновать необходимость всех условий тела. Для  одного из вариантов вопроса и  каждого  варианта задания  2 составить таблицу, отражающую конкретный порядок работы системы.

\subsubsection{Листинг программы}

\lstinputlisting{../src/lab_15.pro}


\subsubsection{Обоснование}

\textit{Максимум из 3 в случае без отсечения}: условия в первом правиле необоходимы для определения, что первое переданное число больше или равно второго числа и больше или равно третьего числа (аналогично остальные два правила).

\textit{Максимум из 3 в случае с отсечением}:

\begin{itemize}
	\item условие первого правила -- для определения, что первое переданное число больше или равно вторму, третьему числу
	\begin{itemize}
		\item если условие выполняется, то прохода дальше не будет, прменяется системный предикат отсечения;
		\item если условие не выполняется, то первое число точно меньше второго или третьего чисел;
	\end{itemize}
	\item условие второго правила (если не выполнилось первое) -- определяется только больше или равно второе число третьему
	\begin{itemize}
		\item если условие выполняется, то прохода дальше не будет, прменяется системный предикат отсечения;
		\item если условие не выполняется, то третье число -- наибольшее из трех чисел.
	\end{itemize}
\end{itemize}

\subsubsection{Выполнение заданий}

Таблицы приложены в конце отчета
